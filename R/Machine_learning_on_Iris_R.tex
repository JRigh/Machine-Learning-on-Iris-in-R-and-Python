\documentclass[border=5mm, convert, usenames, dvipsnames,beamer]{standalone}
\usetheme{Madrid}
\usecolortheme{default}
%Information to be included in the title page:
\title{Sample title}
\author{Anonymous}
\institute{Overleaf}
\date{2021}

\usepackage[absolute,overlay]{textpos}

\defbeamertemplate*{frametitle}{}[1][]
{
    \begin{textblock*}{12cm}(1cm,0.75cm)
    {\color{purple} \fontsize{20}{43.2} \selectfont \insertframetitle}
    \end{textblock*}
    \begin{textblock*}{12cm}(1cm,2.5cm)
    {\color{purple} \fontsize{20}{24} \selectfont \insertframesubtitle}
    \end{textblock*}
}


\usepackage{ragged2e}

\justifying
\usepackage{lmodern}
\usepackage{ImageMagick}
\usepackage[utf8] {inputenc}
\usefonttheme[onlymath]{serif}
\usepackage[english] {label}
\usepackage{amssymb}
\usepackage{amsmath}
\usepackage{bm}
\usepackage{bbm}
\usepackage[round] {natbib}
\usepackage{color}     
\usepackage{changepage}
\usepackage[export]{adjustbox}
\usepackage{graphicx}
\usepackage{minted}
\usepackage{listings}
\usepackage[svgnames]{xcolor}
\usepackage{url}
\urldef\myurl\url{foo%.com}
\usepackage{hyperref}



\setbeamertemplate{footline}[frame number]

\lstdefinestyle{R} {language=R,
    stringstyle=\color{DarkGreen},
    morekeywords={TRUE,FALSE},
    deletekeywords={data,frame,length,as,character},
    keywordstyle=\color{blue},
    commentstyle=\color{teal},    
    basicstyle=\ttfamily\tiny,
    breakatwhitespace=false,         
    breaklines=true,                 
    captionpos=b,                    
    keepspaces=true,                 
    numbers=left,                    
    numbersep=3pt,                  
    showspaces=false,                
    showstringspaces=false,
    showtabs=false,                  
    tabsize=1
}

\definecolor{codegreen}{rgb}{0,0.6,0}
\definecolor{codegray}{rgb}{0.5,0.5,0.5}
\definecolor{codepurple}{rgb}{0.58,0,0.82}

\lstdefinestyle{Python} {language=Python,
    commentstyle=\color{codegreen},
    keywordstyle=\color{magenta},
    numberstyle=\tiny\color{codegray},
    stringstyle=\color{codepurple},
    basicstyle=\ttfamily\tiny,
    breakatwhitespace=false,         
    breaklines=true,                 
    captionpos=b,                    
    keepspaces=true,                 
    numbers=left,                    
    numbersep=3pt,                  
    showspaces=false,                
    showstringspaces=false,
    showtabs=false,                  
    tabsize=1}




\def\one{\mbox{1\hspace{-4.25pt}\fontsize{12}{14.4}\selectfont\textrm{1}}} % 11pt 

\makeatletter
\setbeamertemplate{frametitle}[default]{}
\makeatother
\usepackage{booktabs}
\newcommand{\ra}[1]{\renewcommand{\arraystretch}{#1}}


\begin{document}



\begin{frame}[ fragile]{}
\frametitle{Visualization of Iris dataset}

\vspace{40}
\noindent

\vspace{-5mm}
\begin{figure}[h!]
\begin{center}
\includegraphics[width=11cm]{summaryofdistributions}
\end{center}
\end{figure}







\end{frame}







\begin{frame}[ fragile]{}
\frametitle{Scatterplots with marginal densities}

\vspace{40}
\noindent

\vspace{0mm}
\begin{figure}[h!]
\begin{center}
\includegraphics[width=11cm]{Rplot2}
\end{center}
\end{figure}

\end{frame}




\begin{frame}[ fragile]{}
\frametitle{Some metrics}

\vspace{40}
\noindent

\tiny

$TP =$ true positive, $TN =$ true negative, $FP = $ false positive, $FN =$ false negative

\vspace{10}
\noindent
Accuracy. The number of samples correctly classified out of all the samples present in the (test) set.

$$
Accuracy= \frac{TP + TN}{(TP + TN +FP + FN)}
$$

\vspace{10}
\noindent
Precision (for the positive class). The number of samples actually belonging to the positive class out of all the samples that were predicted to be of the positive class by the model.

$$
Precision = \frac{TP}{(TP +FP)}
$$

\vspace{10}
\noindent
Recall (for the positive class). The number of samples predicted correctly to be belonging to the positive class out of all the samples that actually belong to the positive class.

$$
Recall = \frac{TP}{(TP +NP)}
$$

\vspace{10}
\noindent
F1-Score (for the positive class). The harmonic mean of the precision and recall scores obtained for the positive class.

$$
F1-score = \frac{2 * Precision * Recall}{(Precision + Recall)}
$$



\par



\end{frame}










\begin{frame}[ fragile]{}
\frametitle{Naïve Bayes: confusion matrix}

\vspace{40}
\noindent

\vspace{0mm}
\begin{figure}[h!]
\begin{center}
\includegraphics[width=11cm]{cfnb2}
\end{center}
\end{figure}
\end{frame}




\begin{frame}[ fragile]{}
\frametitle{Naïve Bayes: code chunk and metrics}

\vspace{20}
\noindent


\begin{lstlisting}[style=R]
library(klaR) 

# 1. Build a Naive Bayes Classifier
set.seed(2023)
nb_model <- NaiveBayes(Species ~ ., data=training) # train Naïve Bayes model
pred_nb <- predict(nb_model, testing) # apply Naïve Bayes model on test set
pred_nb_training <- predict(nb_model, training) # apply Naïve Bayes model on train set
\end{lstlisting}


% latex table generated in R 4.2.3 by xtable 1.8-4 package
% Mon Mar 27 13:33:43 2023
\begin{table}[ht]
\centering
\begin{tabular}{rrr}
  \hline
 & Training & Testing \\ 
  \hline
accuracy & 94.90 & 98.08 \\ 
  precision & 94.91 & 97.92 \\ 
  recall & 94.99 & 98.25 \\ 
  f1-score & 94.95 & 98.08 \\ 
   \hline
\end{tabular}
\end{table}






\end{frame}







\begin{frame}[ fragile]{}
\frametitle{Random Forest: confusion matrix}

\vspace{40}
\noindent

\vspace{-5mm}
\begin{figure}[h!]
\begin{center}
\includegraphics[width=11cm]{cmrf2}
\end{center}
\end{figure}






\end{frame}




\begin{frame}[ fragile]{}
\frametitle{Random Forest: code chunk and metrics}

\vspace{20}
\noindent


\begin{lstlisting}[style=R]
library(randomForest)

# 1. Build a Random Forest learning tree Classifier
set.seed(2023)
rf_model <- randomForest(Species~., data=training, ntree=100,proximity=TRUE) # train RF model
pred_rf <- predict(rf_model, testing) # apply RF model on test set
pred_rf_training <- predict(rf_model, training) # apply RF model on train set
\end{lstlisting}


% latex table generated in R 4.2.3 by xtable 1.8-4 package
% Mon Mar 27 13:44:30 2023
\begin{table}[ht]
\centering
\begin{tabular}{rrr}
  \hline
 & Training & Testing \\ 
  \hline
accuracy & 100.00 & 96.15 \\ 
  precision & 100.00 & 96.02 \\ 
  recall & 100.00 & 96.02 \\ 
  f1-score & 100.00 & 96.02 \\ 
   \hline
\end{tabular}
\end{table}







\end{frame}


\begin{frame}[ fragile]{}
\frametitle{Logistic Regression: confusion matrix}

\vspace{40}
\noindent

\vspace{-5mm}
\begin{figure}[h!]
\begin{center}
\includegraphics[width=11cm]{cmlr2}
\end{center}
\end{figure}

\end{frame}




\begin{frame}[ fragile]{}
\frametitle{Logistic Regression: code chunk \\ and metrics}

\vspace{30}
\noindent


\begin{lstlisting}[style=R]
library(stats4) 
library(splines) 
library(VGAM) 

# 1. Build a Multinomial Logistic Regression Classifier
set.seed(2023)
mlr_model <- vglm(Species ~ ., family=multinomial, training)
pred_mlr_training<- predict(mlr_model, training, type = "response")
pred_mlr_testing<- predict(mlr_model, testing, type = "response")
predictions <- apply(pred_mlr_testing, 1, which.max)
predictions_training <- apply(pred_mlr_training, 1, which.max)
\end{lstlisting}


% latex table generated in R 4.2.3 by xtable 1.8-4 package
% Mon Mar 27 14:02:41 2023
\begin{table}[ht]
\centering
\begin{tabular}{rrr}
  \hline
 & Training & Testing \\ 
  \hline
accuracy & 100.00 & 98.08 \\ 
  precision & 100.00 & 98.33 \\ 
  recall & 100.00 & 97.78 \\ 
  f1-score & 100.00 & 98.05 \\ 
   \hline
\end{tabular}
\end{table}

\end{frame}
%------------------------------------------------------------------------------------------






















\begin{frame}[ fragile]{}
\frametitle{Support Vector Machines: \\ confusion matrix}

\vspace{40}
\noindent

\vspace{0mm}
\begin{figure}[h!]
\begin{center}
\includegraphics[width=11cm]{cmsvm2}
\end{center}
\end{figure}




\end{frame}




\begin{frame}[ fragile]{}
\frametitle{Support Vector Machines: \\ code chunk and metrics}

\vspace{30}
\noindent


\begin{lstlisting}[style=R]
library(e1071)

# 1. Build a Support Vector Machines Classifier
set.seed(2023)
svm_model <- svm(Species ~ ., data=training,
                 kernel="radial") #linear/polynomial/sigmoid
pred_svm <- predict(svm_model, testing)
pred_svm_training <- predict(svm_model, training) # apply svm model on train set
\end{lstlisting}


% latex table generated in R 4.2.3 by xtable 1.8-4 package
% Mon Mar 27 14:14:40 2023
\begin{table}[ht]
\centering
\begin{tabular}{rrr}
  \hline
 & Training & Testing \\ 
  \hline
accuracy & 96.94 & 96.15 \\ 
  precision & 97.04 & 96.02 \\ 
  recall & 96.90 & 96.02 \\ 
  f1-score & 96.97 & 96.02 \\ 
   \hline
\end{tabular}
\end{table}


\end{frame}
%------------------------------------------------------------------------------------------




\begin{frame}[ fragile]{}
\frametitle{Neural Network: confusion matrix}

\vspace{40}
\noindent

\vspace{-5mm}
\begin{figure}[h!]
\begin{center}
\includegraphics[width=11cm]{cmnn2}
\end{center}
\end{figure}




\end{frame}




\begin{frame}[ fragile]{}
\frametitle{Neural Networks: \\ code chunk and metrics}

\vspace{40}
\noindent


\begin{lstlisting}[style=R]
library(neuralnet)

# 1. Build a Neural Network Classifier
set.seed(2023)
iris$setosa <- iris$Species=="setosa"
iris$virginica <- iris$Species == "virginica"
iris$versicolor <- iris$Species == "versicolor"

# spliting into test and training again because we added variables
ind <- sample(2, nrow(iris),replace=TRUE,prob=c(0.7,0.3))
training <- iris[ind==1,] ; testing <- iris[ind==2,]

nn_model <- neuralnet(setosa+versicolor+virginica ~ 
                        Sepal.Length + Sepal.Width + Petal.Length + Petal.Width, 
                      data=training, hidden=c(10,10), rep = 5, err.fct = "ce", 
                      linear.output = F, lifesign = "minimal",
                      stepmax = 1000000,  threshold = 0.001)
\end{lstlisting}


% latex table generated in R 4.2.3 by xtable 1.8-4 package
% Mon Mar 27 14:27:47 2023
\begin{table}[ht]
\centering
\begin{tabular}{rrr}
  \hline
 & Training & Testing \\ 
  \hline
accuracy & 100.00 & 98.08 \\ 
  precision & 100.00 & 98.33 \\ 
  recall & 100.00 & 97.78 \\ 
  f1-score & 100.00 & 98.05 \\ 
   \hline
\end{tabular}
\end{table}



\end{frame}
%------------------------------------------------------------------------------------------



\begin{frame}[ fragile]{}
\frametitle{Decision Tree: confusion matrix}

\vspace{40}
\noindent

\vspace{-5mm}
\begin{figure}[h!]
\begin{center}
\includegraphics[width=11cm]{cmdt2}
\end{center}
\end{figure}

\end{frame}



\begin{frame}[ fragile]{}
\frametitle{Decision Tree: code chunk and metrics}

\vspace{40}
\noindent


\begin{lstlisting}[style=R]
library(rpart)

# 1. Build a Decision Tree Classifier
set.seed(2023)
dt_model <- rpart(Species ~.,
                  data = training,
                  method = "class",
                  control = rpart.control(cp = 0),
                  parms = list(split = "information"))

pred_dt_training<- predict(dt_model, training, type = "class")
pred_dt_testing<- predict(dt_model, testing, type = "class")
\end{lstlisting}


% latex table generated in R 4.2.3 by xtable 1.8-4 package
% Mon Mar 27 14:43:16 2023
\begin{table}[ht]
\centering
\begin{tabular}{rrr}
  \hline
 & Training & Testing \\ 
  \hline
accuracy & 100.00 & 100.00 \\ 
  precision & 100.00 & 100.00 \\ 
  recall & 100.00 & 100.00 \\ 
  f1-score & 100.00 & 100.00 \\ 
   \hline
\end{tabular}
\end{table}



\end{frame}
%------------------------------------------------------------------------------------------


\begin{frame}[ fragile]{}
\frametitle{XGBoost: confusion matrix}

\vspace{40}
\noindent

\vspace{-5mm}
\begin{figure}[h!]
\begin{center}
\includegraphics[width=11cm]{cmxgb2}
\end{center}
\end{figure}




\end{frame}




\begin{frame}[ fragile]{}
\frametitle{XGBoost: code chunk and metrics}

\vspace{30}
\noindent


\begin{lstlisting}[style=R]
library(xgboost)

# 0. splitting the dataset into training and test sets
set.seed(2023)
ind <- sample(2, nrow(iris),replace=TRUE,prob=c(0.7,0.3))
training <- iris[ind==1,]
testing <- iris[ind==2,]

xgb_training = xgb.DMatrix(data = as.matrix(training[,-5]), label = training[,5])
xgb_testing = xgb.DMatrix(data = as.matrix(testing[,-5]), label = testing[,5])

# 1. Build a XGBoost Classifier
set.seed(2023)
xgb_model <- xgboost(data=xgb_training, max.depth=3, nrounds=50)

pred_xgb_testing <- predict(xgb_model, xgb_testing)
pred_y_xgb_testing = as.factor((levels(testing[,5]))[round(pred_xgb_testing)])
\end{lstlisting}

% latex table generated in R 4.2.3 by xtable 1.8-4 package
% Mon Mar 27 14:53:37 2023
\begin{table}[ht]
\centering
\begin{tabular}{rrr}
  \hline
 & Training & Testing \\ 
  \hline
accuracy & 100.00 & 100.00 \\ 
  precision & 100.00 & 100.00 \\ 
  recall & 100.00 & 100.00 \\ 
  f1-score & 100.00 & 100.00 \\ 
   \hline
\end{tabular}
\end{table}




\end{frame}
%------------------------------------------------------------------------------------------



\begin{frame}[ fragile]{}
\frametitle{Principal Component Analysis}

\vspace{40}
\noindent

\vspace{0mm}
\begin{figure}[h!]
\begin{center}
\includegraphics[width=11cm]{Rplot10}
\end{center}
\end{figure}




\end{frame}




\begin{frame}[ fragile]{}
\frametitle{K-means}

\vspace{40}
\noindent

\vspace{0mm}
\begin{figure}[h!]
\begin{center}
\includegraphics[width=11cm]{Rplot11}
\end{center}
\end{figure}




\end{frame}




\begin{frame}[ fragile]{}
\frametitle{Partitioning Around Medoids}

\vspace{40}
\noindent

\vspace{0mm}
\begin{figure}[h!]
\begin{center}
\includegraphics[width=11cm]{Rplot12}
\end{center}
\end{figure}




\end{frame}






\begin{frame}[ fragile]{}
\frametitle{Simple Linear Regression}

\vspace{40}
\noindent

\vspace{0mm}
\begin{figure}[h!]
\begin{center}
\includegraphics[width=11cm]{Rplot13}
\end{center}
\end{figure}




\end{frame}







\begin{frame}[ fragile]{}
\frametitle{Simple Log-Linear Regression}

\vspace{40}
\noindent

\vspace{0mm}
\begin{figure}[h!]
\begin{center}
\includegraphics[width=11cm]{Rplot14}
\end{center}
\end{figure}




\end{frame}







\begin{frame}[ fragile]{}
\frametitle{Poisson
 Regression}

\vspace{40}
\noindent

\vspace{0mm}
\begin{figure}[h!]
\begin{center}
\includegraphics[width=11cm]{Rplot15}
\end{center}
\end{figure}




\end{frame}






\begin{frame}[ fragile]{}
\frametitle{Gamma Regression}

\vspace{40}
\noindent

\vspace{0mm}
\begin{figure}[h!]
\begin{center}
\includegraphics[width=11cm]{Rplot16}
\end{center}
\end{figure}




\end{frame}





\begin{frame}[ fragile]{}
\frametitle{Nonparametric Kernel Regression}

\vspace{40}
\noindent

\vspace{0mm}
\begin{figure}[h!]
\begin{center}
\includegraphics[width=11cm]{Rplot17}
\end{center}
\end{figure}




\end{frame}







\begin{frame}[ fragile]{}
\frametitle{Nonparametric Smoothing Splines \\ Regression}

\vspace{40}
\noindent

\vspace{0mm}
\begin{figure}[h!]
\begin{center}
\includegraphics[width=11cm]{Rplot18}
\end{center}
\end{figure}




\end{frame}




\begin{frame}[ fragile]{}

\frametitle{References}

\vspace{-30}
\noindent
The R Project for Statistical Computing:

\vspace{10}
\noindent
\urlf{https://www.r-project.org/}

\vspace{10}
\noindent
https://www.v7labs.com/blog/confusion-matrix-guide


\end{frame}

\end{document}
